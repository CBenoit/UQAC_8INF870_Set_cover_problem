%!TEX root = ./set_cover_problem.tex

%=======================================================================================================
%=============================================== Informations ==========================================
%=======================================================================================================

% Cover infos
\title{Rapport intermédiaire: problème de couverture d'ensemble}
\author{Benoît Cortier \& Maxime Pinard}
\date{\today{}}

% Fancy style options
\lhead{\small Benoît Cortier \& Maxime Pinard}
\rhead{\small Couverture d'ensemble}
\chead{}
\lfoot{}
\rfoot{}
\cfoot{\thepage}
\pagestyle{fancy}

%% Redefine the fancy plain page style
%\fancypagestyle{plain}{
%	\fancyhf{}
%	\lhead{\small Benoît Cortier \& Maxime Pinard}
%	\rhead{\small Couverture d'ensemble}
%	\chead{}
%	\lfoot{}
%	\rfoot{}
%	\cfoot{\thepage}
%}

%=======================================================================================================
%================================================== Configs ============================================
%=======================================================================================================

% Figures folder
\graphicspath{{figures/}}

% Prevent page breaks in paragraphs
\predisplaypenalty=1000
\postdisplaypenalty=1000
\clubpenalty=1000

% Minimal space required in the bottom margin not to move the title on the next page
\renewcommand{\bottomtitlespace}{.1\textheight}

% Links config, especialy for the table of contents
\hypersetup{
    colorlinks=true,
    linkcolor=black,
    urlcolor=blue,
    linktoc=all
}

% French language config
\frenchbsetup{StandardLayout=true,ReduceListSpacing=false,CompactItemize=false}

% Environments
\theoremstyle{definition}
\newtheorem{thm}{Théorème}
\newtheorem{defn}{Définition}
\newtheorem{prop}{Proposition}

% Tables
\newcolumntype{L}[1]{>{\raggedright\let\newline\\\arraybackslash\hspace{0pt}}m{#1}}
\newcolumntype{C}[1]{>{\centering\let\newline\\\arraybackslash\hspace{0pt}}m{#1}}
\newcolumntype{R}[1]{>{\raggedleft\let\newline\\\arraybackslash\hspace{0pt}}m{#1}}
\newcolumntype{Y}{>{\centering\arraybackslash}X}

%=======================================================================================================
%================================================= Functions ===========================================
%=======================================================================================================

% Clear to the next left page
\newcommand*{\cleartoleftpage}{
  \clearpage \ifodd\value{page}\hbox{}\newpage\fi
}

% Paragraph with line break
\newcommand{\p}[1]{\paragraph{#1\\}}

% Function to print a warning sign
\newcommand{\dangersign}[1][2.5ex]
	{\renewcommand{\stacktype}{L}
		{\scaleto{\stackon[1pt]{\color{red}$\triangle$}{\fontsize{4pt}{4pt}\selectfont !}}{#1}}}

% Definition of some dt/dx/dy shortcuts for integrals
\newcommand{\dt}
{\;\mathrm{d}\,t}

\newcommand{\dx}
{\;\mathrm{d}\,x}

\newcommand{\dy}
{\;\mathrm{d}\,y}

% Definition of \Witem for 'itemize' environment with a warning sign
\newcommand{\Witem}
{\item[\dangersign{}]}

% Definition of a Max function shortcut
\newcommand{\Max}[2][ ]
{\underset{#1}{\text{Max}}\,#2}
